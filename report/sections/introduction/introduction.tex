\documentclass[../../main/main.tex]{subfiles}

\begin{document}

\section{Introduction} 

Our best understanding of particle physics is described by the Standard Model that describes the particles that we know and the interactions between them. In 2012 the Higgs particle was discovered by the ATLAS and CMS collaboration at CERN and is the latest particle that was discovered. In 2014 the ATLAS released the Higgs boson machine learning challenge where the objective was to use machine learning to discover the Higgs boson in a dataset consisting of simulated data with background events from a tau tau final state with the signal events that comes from a Higgs boson decaying into the tau tau pair.
In the theory section we describe the theory behind neural networks and how the statistics is used in particle physics to discover particles with the use of a search region and the profile likelihood ration function. Then we explain how a neural network can be used to estimate the profile likelihood ration function which then can be used to calculate the discovery significance. In the method section we describe the Higgs dataset and its challenges with missing variables, and where we propose several methods for how to deal with the missing variables that gives different datasets that the neural network trains on. In the result section we present the output from the neural networks and compare them to each other. There we also implement the two methods for calculating the discover significance with the search region and from the estimation of the profile likelihood ratio function.

 \end{document} 