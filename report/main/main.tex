\documentclass[aps,rmp,reprint,amsmath,amssymb,graphicx,longbibliography]{revtex4-1}

\usepackage{bm}
\usepackage{graphicx}
\usepackage{epstopdf}
\usepackage{wrapfig}
\usepackage{array} 
\usepackage{listings}
\usepackage[para,online,flushleft]{threeparttablex}
\usepackage{booktabs,dcolumn}
\usepackage{color}

\usepackage{ifthen}
% \usepackage{bbding}
\usepackage{diagbox}
\usepackage{pifont}
\usepackage{wasysym}
\usepackage{amssymb} 

\usepackage{bm}

% \usepackage{multicol}

\usepackage{caption}
\usepackage{subcaption}
\usepackage{listofitems}

\usepackage{tikz}

\usepackage{enumitem}
\usepackage{textpos}
\usepackage{booktabs}
\usepackage{multirow,bigdelim}
\usepackage{float}

\usepackage{upgreek} %upalpha in Saxena2021 Reference

\usepackage[utf8]{inputenc}

\usepackage{hyperref}
\hypersetup{breaklinks=true,colorlinks=true,linkcolor=blue,citecolor=blue,filecolor=magenta,urlcolor=blue}




\usepackage{xcolor}
%\newcommand{\contrib}[1]{\textcolor{red}{#1}}
%\newcommand{\comment}[1]{\textcolor{blue}{#1}}

%\newcommand{\WN}[1]{{\color{red} #1}}

\makeatletter
\def\@bibdataout@aps{%
\immediate\write\@bibdataout{%
@CONTROL{%
apsrev41Control%
\longbibliography@sw{%
    ,author="08",editor="1",pages="1",title="0",year="1"%
    }{%
    ,author="08",editor="1",pages="1",title="",year="1"%
    }%
  }%
}%
\if@filesw \immediate \write \@auxout {\string \citation {apsrev41Control}}\fi
}
\makeatother

\usepackage{subfiles}
%\usepackage[left]{lineno}
%\linenumbers


\begin{document}
%\pagenumbering{gobble} 

\title{Training Neural Networks for Likelihood Ratio Estimation for the Higgs Boson Machine Learning Challenge with Various Strategies to Account for Missing Variables}

\author{Martin Føll}
\affiliation{Department of Physics, University of Oslo, N-0316 Oslo, Norway}


\begin{abstract}
  In this project we do the Higgs boson machine learning challenge \cite{pmlr-v42-cowa14} to discover the Higgs boson using a neural network, and to calculate the discover significance with the method of a search region and from the profil likelihood function that is estimated from the neural network. The dataset used for the challenge was used in different ways to train the neural network by inserting different values into the dataset for missing values, removing features with missing variables and splitting the dataset into three disjoint datasets which dependend on the number of jets in the events. The dataset that gave the highest discovery significance was the \emph{FillMean} dataset which gave \(Z=429.963\) from the likelhood estimation. However, the \emph{JetsTwo} dataset gave the highest AUC equal to 0.932, but because of less statistics it gave lower discovery significance than \emph{FillMean}.
\end{abstract}

\maketitle

\tableofcontents


\subfile{../sections/introduction/introduction}
\subfile{../sections/theory/theory}
\subfile{../sections/method/method}
\subfile{../sections/results/results}

\appendix
\begin{align}
  -2\ln \lambda(\mu) &= -2\ln\left(\frac{L_{s+b}(\mu=0,\hat{\hat{\bm{\theta}}})}{L_{s+b}(\mu=1,\hat{\hat{\bm{\theta}}})}\right) \notag \\
             &= -2\ln\frac{L_{s+b}(\mu,\hat{\hat{\bm{\theta}}})}{L_{s+b}(\mu=1,\hat{\hat{\bm{\theta}}})} \notag \notag \\
             &= -2\ln L_{s+b}(\mu,\hat{\hat{\bm{\theta}}}) - 2\ln L_{s+b}(\mu=1,\hat{\hat{\bm{\theta}}}) \notag \\              &= -2\ln L_{s+b}(\mu,\hat{\hat{\bm{\theta}}}) - 2\ln L_{s+b}(\mu=1,\hat{\hat{\bm{\theta}}}) \notag \\              &+2\ln L_b(\hat{\hat{\bm{\theta}}}) - 2\ln L_b(\hat{\hat{\bm{\theta}}}) \notag \\
             &= -2\ln \frac{L_{s+b}(\mu,\hat{\hat{\bm{\theta}}})}{L_b(\hat{\hat{\bm{\theta}}})} - 2\ln \frac{L_{s+b}(\mu=1,\hat{\hat{\bm{\theta}}})}{L_b(\hat{\hat{\bm{\theta}}})} \notag \\
             &= -2\ln Q(\mu) + 2\ln Q(1) \notag \\
             &= t(\mu) - t(1),   \label{eq:14}
\end{align}

\cite{moustakides2019training}
\bibliography{References} % add  references to this file
\end{document}
